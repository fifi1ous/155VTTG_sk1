\section{Postup zpracování}

\subsection{Nivelační měření}

Převýšení mezi přestavovými body bylo spočteno přímo digitálním nivelačním přístrojem. Ten vyhodnotil výsledné převýšení na základě čtyřnásobného odečtu v pořadí BFFB (zadní – přední – přední – zadní) podle vzorce:
\[
\Delta h = \frac{(B_1 + B_2) - (F_1 + F_2)}{2}
\]
kde \( B_1, B_2 \) jsou zadní odečty a \( F_1, F_2 \) přední odečty.\\
V rámci zpracování byla provedena kontrola přesnosti měření porovnáním rozdílu převýšení mezi směrem „tam“ a „zpět“ v každém oddílu. Tento rozdíl byl testován vůči mezní hodnotě přesnosti dle ČSN pro VPN, která je dána vztahem \cite{skorepa}:
\[
\Delta_M = 2,25 \cdot \sqrt{R}
\]
kde \( R \) je délka oddílu v kilometrech.\\
Dále byl pro každý oddíl vypočten aritmetický průměr převýšení:
\[
\Delta h_{\diameter} = \frac{\Delta h_{\text{TAM}} + \Delta h_{\text{ZPĚT}}}{2}
\]
V případě, že rozdíl převýšení ve směrech „tam“ a „zpět“ překročil tuto mezní hodnotu, bylo měření vyhodnoceno jako nevyhovující.\cite{zadani}\\
Pro posouzení přesnosti nivelace se počítá směrodatná odchylka kilometrová
\[
  m_0 = \frac{1}{2}\,\sqrt{\frac{\sum\rho^2}{R}}
\]
kde \( R \) je délka oddílu v km a \(\rho\) rozdíly převýšení „tam“ vs. „zpět“ v jednotlivých oddílech.\\
Směrodatná odchylka celého pořadu podle délky [km]
\[
  m_h = m_0 \,\sqrt{L}
\]
kde \(L\) je celková délka pořadu v kilometrech.\\
Mezní kilometrová směrodatná odchylka pro II. řád
\[
  \overline{m_0} = 0{,}45 + \frac{0{,}80}{\sqrt{n}}\quad [\mathrm{mm}]
\]
kde \(n\) je počet oddílů nivelace.

\subsection{Tíhové měření}

Naměřená data (čas \(T_i\), vnitřní teplota \(t_i\), čtení přístroje \(S_i\)) byla zpracována v těchto krocích \cite{gravi}:

\begin{itemize}
  \item \textbf{Přepočet dílků přístroje na mGal:}
    \[
      gr''_i = C\,S_i,
      \quad C = 4{,}379\;\text{mGal/dílek}
    \]
  \item \textbf{Výpočet chodu přístroje:}
    \[
      \delta g_{\mathrm{drift},i}
        = \frac{\overline{gr''}_{\mathrm{end}} - \overline{gr''}_{\mathrm{start}}}
               {T_{\mathrm{end}} - T_{\mathrm{start}}}
          \,\bigl(T_i - T_{\mathrm{start}}\bigr)
    \]
  \item \textbf{Aplikace korekce z chodu:}
    \[
      gr'_i = gr''_i + \delta g_{\mathrm{drift},i}
    \]
  \item \textbf{Výpočet rozdílů tíhového zrychlení mezi sousedními body:}
    \[
      \delta g_{AB} = g_B - g_A
    \]
  \item \textbf{Výpočet absolutního tíhového zrychlení:}
    \[
      g_i = g_{\mathrm{ref}}
          + \bigl(gr'_i - \overline{gr''}_{\mathrm{start}}\bigr)
    \]
    kde \(g_{\mathrm{ref}}\) je referenční hodnota na bodě 3408.01. Přičtením \(\delta g_{AB}\) k referenční hodnotě se získají absolutní hodnoty \(g_i\) pro ostatní body.
\end{itemize}


\subsection{Výpočet Normálních (Moloděnského) výšek}

Přibližné výšky bodů byly vypočítány pomocí nivelovaného převýšení (h):

\[
H_n = H_1+\sum_{i=1}^n h_i
\]
Kde \( H_n \) představuje kumulativní součet výšek jednotlivých bodů \( h \).
\subsection*{Normální ortometrická korekce \( c_{\gamma AB} \)}
Normální ortometrická korekce byla vypočtena jako: 
\[
c_{\gamma AB} = -0.0000254 \cdot H_s \cdot \Delta B \cdot 0.001
\]
kde \(H_s\) je střední výška mezi dvěma body a \(\Delta B\) je rozdíl zeměpisné šířky.
\subsection*{Korekce z tíhových anomálií \( c_{\Delta g AB} \)}
Korekce z tíhových anomálií byla vypočítána podle následujícího postupu:

\begin{itemize}
    \item \textbf{Výpočet normálního tíhového zrychlení z Helmertova vzorce}: 
    \[
    \gamma_0 = 978030 \left(1 + 0.005302 \cdot \sin^2(B) - 0.000007 \cdot \sin^2(2B) \right) \cdot 10^{-5}
    \]
    \item \textbf{Výpočet Fayovy tíhové anomálie}:
    \[
    \Delta g_F = g + 0.3086 \cdot 10^{-5} \cdot H - \gamma_0
    \]
    \item \textbf{Průměrná tíhová anomálie mezi nivelovanými body}:
    \[
    \Delta g_{F_{AB}} = \frac{\Delta g_F{_{1}} + \Delta g_{F_{2}}}{2}
    \]
    \item \textbf{Korekce z tíhové anomálie}:
    \[
    c_{\Delta g AB} = 0.0010193 \cdot 10^5 \cdot \Delta g_{F_{AB}} \cdot H_{\text{niv}} \cdot 0.001
    \]
\end{itemize}

\subsection*{Výpočet normální Moloděnského výšky}
Po výpočtu všech korekcí byla normální převýšení určena podle vztahu:
\[
h_Q = H_{2} + c_{\gamma AB} + c_{\Delta g AB}
\]
A celková normální výška byla získána kumulativním součtem všech korekcí:
\[
H_Q = H_{2} + \sum c_{\gamma AB} + \sum c_{\Delta g AB}
\]

\subsection{Výpočet Bouguerovy anomálie \(B_a\)}
Do nivelačních údajů každého výškového bodu bylo třeba uvést kromě absolutního a normálního tíhového zrychlení ještě hodnotu Bouguerovy anomálie v Postupimské tíhové soustavě, která byla vypočtena jako:
\[
B_a = g - \gamma_0 + (0.3086 - 0.1119)\cdot H_Q + 14 mGal
\]

\subsection{Sestavení nivelačních údajů}
Na závěr celého zpracování byly vytvořeny nivelační údaje bodů pomocí formuláře, u bodů 33.1 a 34 aktualizované, pro nové body 35.1 a 36.1 nové údaje. Formuláře nivelačních údajů byly vyplněny dle vzoru \cite{skorepa}. V údajích byly obsaženy délky oddílů a celkového pořadu od počátku, nová nadmořská výška bodů (u bodu 33.1 převzatá ze současně platných údajů), místopisný náčrt a popis bodu a jeho okolí, údaje o umístění v podrobnosti od okresu až do parcelního čísla včetně identifikace vlastníka parcely, druh nivelační značky, stupeň, druh a autor stabilizace bodu, přibližné souřadnice v S-JTSK, přibližná zeměpisná délka a šířka a tíhové údaje, tedy absolutní a normální tíhové zrychlení a Bouguerova anomálie.