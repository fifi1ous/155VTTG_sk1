\section{Postup zpracováni}

\subsection{Nivelační měření}

Převýšení mezi přestavovými body bylo spočteno přímo digitálním nivelačním přístrojem. Ten vyhodnotil výsledné převýšení na základě čtyřnásobného odečtu v pořadí BFFB (zadní – přední – přední – zadní) podle vzorce:
\[
\Delta h = \frac{(B_1 + B_2) - (F_1 + F_2)}{2}
\]
kde \( B_1, B_2 \) jsou zadní odečty a \( F_1, F_2 \) přední odečty.\\
V rámci zpracování byla provedena kontrola přesnosti měření porovnáním rozdílu převýšení mezi směrem „tam“ a „zpět“ v každém oddílu. Tento rozdíl byl testován vůči mezní hodnotě přesnosti dle ČSN pro VPN, která je dána vztahem:
\[
\Delta_M = 2,25 \cdot \sqrt{R}
\]
kde \( R \) je délka oddílu v kilometrech.\\
Dále byl pro každý oddíl vypočten aritmetický průměr převýšení:
\[
\Delta h_{\diameter} = \frac{\Delta h_{\text{TAM}} + \Delta h_{\text{ZPĚT}}}{2}
\]
V případě, že rozdíl převýšení ve směrech „tam“ a „zpět“ překročil tuto mezní hodnotu, bylo měření vyhodnoceno jako nevyhovující.\cite{zadani}

\subsection{Tíhové měření}
