\section{Postup měření}

\subsection{Nivelační měření}

Měření bylo realizováno metodou velmi přesné nivelace (VPN) na části úseku nivelačního pořadu II. řádu v okolí Nové Seninky. Měřická skupina byla rozdělena na dvě samostatné čety, přičemž každá z nich samostatně provedla měření jednoho oddílu v obou směrech – tam i zpět. Třetí oddíl byl zaměřen společně tak, že směr „tam“ provedla první četa a směr „zpět“ druhá četa. Každá měřická četa pracovala s vlastním nivelačním přístrojem Leica Wild NA3003 a dvojicí nivelačních latí s čárovým kódem. Převýšení bylo určováno ze čtení v pořadí zadní – přední – přední – zadní (BFFB). V případě překročení mezní hodnoty rozdílu mezi dvěma nezávislými výpočty převýšení bylo měření příslušného oddílu opakováno. Záměry byly ve strmějších úsecích drženy maximálně do 20 m, v rovinatějších úsecích až do 40 m, přičemž minimální výška záměry nad terénem byla kontrolována tak, aby neklesla pod 40 cm. Všechny přestavové body byly stabilizovány pomocí kovových nivelačních hřebů. Vzdálenosti mezi přestavovými body byly rozměřovány pomocí měřických koleček. Naměřená data byla uložena ve formátu GSI do paměťových modulů přístroje a následně exportována pro další zpracování.

\subsection{Tíhové měření}

Gravimetrické měření bylo provedeno sovětským gravimetrem GAK s konstantou $C = 4,379\,$mGal/díl. Měření bylo provedeno po třech dvojicích, kdy každá dvojice prováděla měření na výchozím tíhovém bodu 3408.01, jež je pro účely úlohy totožný s bodem 33.1. Dále bylo měření provedeno na bodech: 34, 35.1, 36.1 a nakonec opět na připojovacím tíhovém bodě 3408.01, aby mohl být určen chod gravimetru. Výsledky všech tří skupin jsou nakonec průměrovány. V průběhu měření byly kromě přístrojem měřené hodnoty zapisovány i teplota uvnitř přístroje a časy jednotlivých čtení. 