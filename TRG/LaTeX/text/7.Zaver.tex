\newpage
\section{Závěr}
\tab V rámci zpracování byla nejprve vyhodnocena centrační osnova, z níž byly stanoveny průměry a opraveny délky o součtovou konstantu hranolu. Následně proběhla redukce délek, zahrnující fyzikální redukce podle meteorologických veličin a matematické redukce zohledňující refrakci a převod do roviny Křovákova zobrazení. Pro astronomické měření byly vypočteny průměry směrů na levý a pravý okraj Slunce i odpovídající časy, čímž byl získán směr ke středu slunečního kotouče a z tabulek určena rektascenze a deklinace pro výpočet azimutu. Současně byly stanoveny směrové korekce do rovinného zobrazení a iteračním postupem byly určeny centrační změny vodorovných směrů na všech stanoviscích. Dále proběhlo vyrovnání sítě v programu GAMA, kde byly pro vyrovnání použito 15 směrů, 15 délek a 20 azimutů. Výsledné vyrovnané souřadnice jsou patrné v následující tabulce.
\begin{table}[H]
\centering
\caption{Vyrovnané souřadnice bodů.}
\begin{tabular}{c S[table-format=7.4] S[table-format=7.4]}
\toprule
\textbf{Bod} & {$\mathbf{X}$ \textbf{[m]}} & {$\mathbf{Y}$ \textbf{[m]}} \\
\midrule
1001 & 1055386.1209 & 565725.1145 \\
1002 & 1058509.5130 & 560713.0936 \\
1003 & 1055296.0630 & 560933.3479 \\
1004 & 1053013.6070 & 559765.5664 \\
\bottomrule
\end{tabular}
\label{tab:adjusted_coords_vse}
\end{table}
\tab Z dostupných RINEX dat byly vypočítány souřadnice z GNSS, které byly porovnány s terestrickým měřením. Porovnání je uvedeno v následující tabulce.
\begin{table}[H]
\centering
\caption{Souřadnice a jejich rozdíly v metrech}
\begin{tabular}{|c||c|c|c||c|c|c|}
\hline
\textbf{Stanovisko} & \textbf{X vyrovnané} & \textbf{X GNSS} & $\mathbf{\Delta X}$ & \textbf{Y vyrovnané} & \textbf{Y GNSS} & $\mathbf{\Delta Y}$ \\
\hline\hline
1001 & 1055386.121 & 1055386.291 & 0.170 & 565725.114 & 565725.159 & 0.045 \\ \hline
1002 & 1058509.513 & 1058509.547 & 0.034 & 560713.094 & 560713.097 & 0.003 \\ \hline
1003 & 1055296.063 & 1055296.051 & -0.012 & 560933.348 & 560933.350 & 0.002 \\ \hline
1004 & 1053013.607 & 1053013.574 & -0.033 & 559765.566 & 559765.560 & -0.006 \\
\hline
\end{tabular}
\end{table}
\tab Dalším výstupem je určení konstanty gyroteodolitu, která činí \textbf{7,1844}.

