\section{Postup měření}
\tab Měření v terénu probíhalo ve dvou dnech a zahrnovalo klasické geodetické metody – triangulaci, trilateraci a astronomické určení azimutu, doplněné o měření gyroteodolitem. Studenti byli rozděleni do čtyř pracovních čet, přičemž každá z nich zaujala jeden z určených bodů sítě a během celého dne prováděla veškerá potřebná měření. Pro práci byly využity univerzální teodolity typu Leica TC1700 nebo Topcon GPT-7501.\\
\tab %Při triangulačních a trilateračních měřeních se na každém bodě určovaly vodorovné úhly, a to opakovaně ve více sadách, aby byla zajištěna spolehlivost výsledků. Vzhledem k excentrickému uspořádání stanoviska a cíle byla součástí práce také centrace, tedy určení vzájemných vazeb mezi centrem, excentrickým stanoviskem a cílem. Současně se měřily vzdálenosti na ostatní body sítě. Délky se vždy určovaly ze stanoveného centra a spolu s tím byly zaznamenávány meteorologické veličiny – teplota, tlak a vlhkost vzduchu – nutné pro následnou fyzikální redukci. Důležitá byla i evidence výšek přístroje a odrazných hranolů nad body. %Aby bylo možné navázat měření mezi jednotlivými četami, každá skupina po skončení dne provedla předzpracování svých dat a výsledky zpřístupnila v dohodnutém formátu ostatním.\\
V rámci triangulace a trilaterace byly na každém stanovisku měřeny tři vodorovné úhly. Každý úhel byl určen nezávisle ve třech sadách, přičemž v každé sadě se uskutečnilo dvojí cílení, aby bylo dosaženo vyšší přesnosti. Délky se vždy určovaly ze stanoveného centra a spolu s tím byly zaznamenávány meteorologické veličiny – teplota, tlak a vlhkost vzduchu – nutné pro následnou fyzikální redukci. Důležitá byla i evidence výšek přístroje a odrazných hranolů nad body. Důležitou součástí postupu bylo měření centračních prvků, protože přístroje i cíle byly umístěny excentricky. Centrační osnova obsahovala směry na tři ostatní body sítě a navíc na vlastní centr a excentrický cíl. Dále byly změřeny délky z excentrického stanoviska na centr a na excentrický cíl. \\
\tab Astronomické určení azimutu spočívalo v měření směru na Slunce. Každá četa si zvolila jednu stranu vymezenou excentrickým stanoviskem a cílem a provedla měření úhlu mezi touto záměrou a polohou Slunce. Postup zahrnoval cílení na oba okraje slunečního kotouče v obou polohách dalekohledu, přičemž byl vždy zaznamenán přesný čas měření. K určení času se využíval ruční GPS přijímač nastavený na světový čas UTC, zatímco vlastní odečet probíhal pomocí stopek synchronizovaných s tímto přijímačem. Získaná data pak sloužila k výpočtu azimutu měřené strany.\\
\tab Další část měření byla věnována práci s gyroteodolitem. Jeho úkolem bylo určení azimutu vybrané strany v síti. Protože vlastní součtová konstanta použitého přístroje nebyla známa, postup byl obrácený: z naměřených hodnot a známého azimutu se zpětně určovala právě tato konstanta.

