\section{Výsledky}


\begin{itemize}
    \item Tabulky dokumentující redukce měřených délek (fyzikální, matematická redukce). 
          Pro redukce délek použijete výšky získané pomocí GNSS.
    \item Tabulky dokumentující redukce měřených úhlů (centrace, směrové korekce).
    \item Tabulky dokumentující určení azimutu jednotlivých stran.
    \item Popis způsobu stanovení vah měřených veličin vstupujících do vyrovnání.
    
    \item Porovnání výsledků získaných z terestrických měření a pomocí GNSS. 
          Posouzení rozdílů (rozdíly v rozměru sítě, orientaci, \ldots).
    
    \item Určená hodnota konstanty gyroteodolitu.
\end{itemize}



\begin{table}[h!]
\centering

\label{tab:gnss_coords}
\begin{tabular}{lccc}
\hline
stanovisko & Zeměpisná šířka [°] & Zeměpisná délka [°] & Výška [m] \\
\hline
1001 & 50.168009 & 16.900870 & 877.078 \\
1002 & 50.144727 & 16.975136 & 746.024 \\
1003 & 50.173260 & 16.967453 & 775.968 \\
1004 & 50.194748 & 16.980437 & 957.596 \\
1005 & 50.168051 & 16.946863 & 602.168 \\
\hline

\end{tabular}
\caption{Souřadnice stanovišť určené z GNSS měření}
\end{table}
