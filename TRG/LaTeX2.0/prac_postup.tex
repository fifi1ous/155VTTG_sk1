
\subsubsection{Centrační osnova}
\tab Každou z měřických čet bylo nejprve zpracováno měření centrační osnovy. Byly tedy vypočteny
průměry ze dvou skupin, měřené délky byly opraveny o součtovou konstantu podle použitého
hranolu.


\subsubsection{Redukce délek}



\subsubsection{Astronomické azimuty}
\tab Nejprve byl vypočten průměr z měření na levý a pravý okraj Slunce a také průměr z časů těchto
měření. Výpočtem průměru byl vodorovný směr vztažen ke středu Slunce. Díky tomu mohla být
dále použita rektascenze a deklinace v astronomických tabulkách. Následně byl jako rozdíl dvou směrů vypočten vodorovný úhel mezi bodem sítě a středem Slunce.


\subsubsection{Redukce měřených směrů do Křovákova zobrazení}



\subsubsection{Centrace vodorovných směrů}
\tab Centrace byla provedena iteračním postupem. Zvlášť byla určována centrační změna pro každé
z ramen měřeného úhlu.

Pro každé rameno byla zvolena místní soustava souřadnic, jejíž počátek ležel na centrickém sta-
novisku (T1) a kladná osa +x směřovala do centrického cíle (T2). Vzhledem k velké vzdálenosti
mezi oběma body mohla být v první iteraci spojnice excentrického stanoviska (S1) s excentrickým
cílem (C2) považována za totožnou s osou x a tedy její směrník $\sigma_{S1,C2} = 0$. Přičtením příslušného úhlu měřeného v rámci centrační osnovy lze určit směrník mezi excentrickým stanoviskem (S1) a centrickým stanoviskem (T1). Jeho obrácením o 200 gon lze pak při známé hodnotě excentricity
určit rajonem souřadnice excentrického stanoviska (S1) v místní soustavě. Opětovným přičtením
příslušného úhlu měřeného v centrační osnově lze určit směrník mezi excentrickým stanoviskem
(S1) a excentrickým cílem (C1) na témže bodě. Souřadnice excentrického cíle (C1) se při známé
excentricitě opět určí rajonem. Obdobný postup byl aplikován na bodě 2 a byly tak určeny sou-
řadnice S2, C2. Ze souřadnic v místní soustavě mohl být nově spočten směrník $\sigma_{S1,C2}$ , který byl již různý od 0. Následně byl celý postup opakován a to do té doby, dokud změna směrníku nebyla
menší než 0,00001 gon. Centrační změna $\delta^1$ směru ramene je pak rovna přímo hodnotě směrníku
$\sigma_{S1,C2}$.



\subsubsection{Vyrovnání sítě}
