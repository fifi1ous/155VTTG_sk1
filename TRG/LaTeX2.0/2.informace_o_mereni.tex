
%\setstretch{1.5} % řádkování
\begin{tabular}{lll} 
\textit{Místo měření:} & & Staré Město pod Sněžníkem a okolí (okres Šumperk)\\ 
\textit{Datum měření:} & & 13.~6.~2025 – triangulace, trilaterace, astro měření;\\
& & 16.~6.~2025 – měření gyroteodolitem\\
\textit{Povětrnostní podmínky:} & & 13.~6.~2025 – jasno, slabý vítr, teplota cca 20–24°C\\
 & & 16.~6.~2025 – zataženo, deštivo,  teplota cca 17–20~°C.\\
\textit{Použité přístroje a pomůcky:} & & 2× totální stanice Leica TC1700 / Topcon GPT‑7501,\\
& & souprava hranolů (centrické a excentrické), minihranol,\\
& & 2× stativ, 2× měřická lať,\\
& & gyroteodolit, stopky, přijímač pro čas UTC,\\
& & meteorologická souprava (teploměr, vlhkoměr, barometr)\\
\textit{Souřadnicový systém:} & & S‑JTSK\\ 
\end{tabular}
