
Cílem této úlohy je určení souřadnic vybraných bodů geodetické sítě v okolí Starého Města pod Sněžníkem v souřadnicovém systému~S‑JTSK pomocí metod klasické geodézie – \textbf{triangulace}, \textbf{trilaterace} a \textbf{astronomického určení azimutu}. Dalším cílem je stanovení \textbf{součtové konstanty gyroteodolitu} z měření na známém azimutu. Orientace sítě je definována měřenými azimuty na vybraných bodech. Vzhledem k obtížnosti určení zeměpisných souřadnic bodů sítě astronomickými metodami jsou souřadnice jednoho z bodů převzaty z úlohy GNSS. V rámci úlohy jsou prováděny následující činnosti:
\begin{itemize}
    \item měření horizontálních směrů mezi body (triangulace) včetně centrace měřených směrů.
    \item měření délek mezi body (trilaterace) a záznam meteorologických veličin pro fyzikální redukce.
    \item provedení astronomického určení azimutu na vybraných stranách sítě měřením na Slunce.
    \item měření gyroteodolitem na vybrané straně a výpočet součtové konstanty přístroje.
    \item zpracování a vyrovnání výsledků, porovnání s výsledky GNSS.
\end{itemize}
