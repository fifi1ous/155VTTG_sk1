\tab Určete průběh kvazigeoidu v profilu vedeném podél nivelačního pořadu II.řádu Z7ab Žleb-Kunčice pomocí GNSS-nivelace. Dosažené výsledky budou použity k otestování modelu kvazigeoidu CR2005 v dané oblasti.\\

Na profilu realizovaném dočasně stabilizovanými body a vedeném podél nivelačního pořadu Z7ab (úsek Vysoké Žibřidovice - Kladské sedlo vymezený body Z7ab-9 a Z7ab-44) určete pro jednotlivé body profilu elipsoidické souřadnice (polohu a výšku) v systému ETRS89 (realizace ETRF2000) a normální výšku v systému Bpv. Elipsoidické souřadnice zaměřte pomocí technologie GNSS statickou metodou. Normální výšku bodů profilu určete technickou nivelací z blízkých bodů nivelačního pořadu Z7ab.\\

V úseku Nová Seninka - Kladské sedlo (úsek nivelačního pořadu Z7ab vymezený body Z7ab-33 a Z7ab-44) bude GNSS-nivelace navázána na prováděnou obnovu nivelačního pořadu Z7ab (úloha VPN). V tomto úseku využijte nově určené normální výšky připojovacích nivelačních bodů z provedené obnovy pořadu. Ve zbylé části profilu (úsek Vysoké Žibřidovice - Nová Seninka) převezměte normální výšky bodů z nivelačních údajů.\\

Získané hodnoty výškové anomálie kvazigeoidu porovnejte pro body vašeho profilu s hodnotami modelu kvazigeoidu CR2005.\\