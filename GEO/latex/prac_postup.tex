
\subsection{Měření GNSS}

\tab Měření GNSS probíhalo rychlou statickou metodou s použitím referenční stanice ve Starém Městě (TSTA). Data z měření byla exportována do formátu RINEX. Byla použita observační data ze stanice CZEPOS v Šumperku (CSUM).

Zpracování bylo provedeno v programu RTKLib. 

\subsection{Technická nivelace}

\tab Nivelace byla použita pro určení výšky bodu z excentrického stanoviska, s maximem dvou přestav.

Hodnocení přesnosti nivelace porovnáním rozdílů nivelovaných převýšení a mezní odchylkou probíhalo už v terénu podle:

\begin{equation}
    \Delta M_{[mm]}=0,67*40*\sqrt{R_{[km]}},
\end{equation}

kde $R_{[km]}$ je délka pořadu v kilometrech.\\

Vyhovující dvojice měření pak byla zprůměrována.\\

Přesnost měření popisuje střední jednotková chyba kilometrová obousměrné nivelace:

\begin{equation}
    m_{0[mm]}=\frac{1}{2}\sqrt{\frac{\delta h}{R_{[km]}}},
\end{equation}

směrodatná odchylka nivelovaného převýšení:

\begin{equation}
    m_{[mm]}=m_0\sqrt{R_{[km]}},
\end{equation}

a její mezní hodnota:

\begin{equation}
    m_{[mm]}=1,00 + \frac{1,77}{\sqrt{M}},
\end{equation}

kde $M$ je počet oddílů v pořadu.

Výšky měřené dvěma skupinami byly zprůměrovány a ze zákona hromadění směrodatných odchylek je směrodatná odchylka průměru:

\begin{equation}
    \sigma_{\varnothing m}=2\sqrt{\sigma_{m1}^2+\sigma_{m2}^2}.
\end{equation}

\subsection{Výpočet výškové anomálie}

\tab Výšková anomálie je rozdíl elipsoidické (určené GNSS) a normální Moloděnského výšky (určené nivelací): 

\begin{equation}
    N=H_{el}-H_{niv}.
\end{equation}

Její přesnost je ze zákona hromadění směrodatných odchylek:

\begin{equation}
    \sigma_N=\sqrt{\sigma_{Hel}^2+\sigma_{Hniv}^2}.
\end{equation}
